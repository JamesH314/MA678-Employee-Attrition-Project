% Options for packages loaded elsewhere
\PassOptionsToPackage{unicode}{hyperref}
\PassOptionsToPackage{hyphens}{url}
%
\documentclass[
]{article}
\usepackage{amsmath,amssymb}
\usepackage{lmodern}
\usepackage{ifxetex,ifluatex}
\ifnum 0\ifxetex 1\fi\ifluatex 1\fi=0 % if pdftex
  \usepackage[T1]{fontenc}
  \usepackage[utf8]{inputenc}
  \usepackage{textcomp} % provide euro and other symbols
\else % if luatex or xetex
  \usepackage{unicode-math}
  \defaultfontfeatures{Scale=MatchLowercase}
  \defaultfontfeatures[\rmfamily]{Ligatures=TeX,Scale=1}
\fi
% Use upquote if available, for straight quotes in verbatim environments
\IfFileExists{upquote.sty}{\usepackage{upquote}}{}
\IfFileExists{microtype.sty}{% use microtype if available
  \usepackage[]{microtype}
  \UseMicrotypeSet[protrusion]{basicmath} % disable protrusion for tt fonts
}{}
\makeatletter
\@ifundefined{KOMAClassName}{% if non-KOMA class
  \IfFileExists{parskip.sty}{%
    \usepackage{parskip}
  }{% else
    \setlength{\parindent}{0pt}
    \setlength{\parskip}{6pt plus 2pt minus 1pt}}
}{% if KOMA class
  \KOMAoptions{parskip=half}}
\makeatother
\usepackage{xcolor}
\IfFileExists{xurl.sty}{\usepackage{xurl}}{} % add URL line breaks if available
\IfFileExists{bookmark.sty}{\usepackage{bookmark}}{\usepackage{hyperref}}
\hypersetup{
  pdftitle={Midterm Project},
  pdfauthor={Yifeng He},
  hidelinks,
  pdfcreator={LaTeX via pandoc}}
\urlstyle{same} % disable monospaced font for URLs
\usepackage[margin=1in]{geometry}
\usepackage{color}
\usepackage{fancyvrb}
\newcommand{\VerbBar}{|}
\newcommand{\VERB}{\Verb[commandchars=\\\{\}]}
\DefineVerbatimEnvironment{Highlighting}{Verbatim}{commandchars=\\\{\}}
% Add ',fontsize=\small' for more characters per line
\usepackage{framed}
\definecolor{shadecolor}{RGB}{248,248,248}
\newenvironment{Shaded}{\begin{snugshade}}{\end{snugshade}}
\newcommand{\AlertTok}[1]{\textcolor[rgb]{0.94,0.16,0.16}{#1}}
\newcommand{\AnnotationTok}[1]{\textcolor[rgb]{0.56,0.35,0.01}{\textbf{\textit{#1}}}}
\newcommand{\AttributeTok}[1]{\textcolor[rgb]{0.77,0.63,0.00}{#1}}
\newcommand{\BaseNTok}[1]{\textcolor[rgb]{0.00,0.00,0.81}{#1}}
\newcommand{\BuiltInTok}[1]{#1}
\newcommand{\CharTok}[1]{\textcolor[rgb]{0.31,0.60,0.02}{#1}}
\newcommand{\CommentTok}[1]{\textcolor[rgb]{0.56,0.35,0.01}{\textit{#1}}}
\newcommand{\CommentVarTok}[1]{\textcolor[rgb]{0.56,0.35,0.01}{\textbf{\textit{#1}}}}
\newcommand{\ConstantTok}[1]{\textcolor[rgb]{0.00,0.00,0.00}{#1}}
\newcommand{\ControlFlowTok}[1]{\textcolor[rgb]{0.13,0.29,0.53}{\textbf{#1}}}
\newcommand{\DataTypeTok}[1]{\textcolor[rgb]{0.13,0.29,0.53}{#1}}
\newcommand{\DecValTok}[1]{\textcolor[rgb]{0.00,0.00,0.81}{#1}}
\newcommand{\DocumentationTok}[1]{\textcolor[rgb]{0.56,0.35,0.01}{\textbf{\textit{#1}}}}
\newcommand{\ErrorTok}[1]{\textcolor[rgb]{0.64,0.00,0.00}{\textbf{#1}}}
\newcommand{\ExtensionTok}[1]{#1}
\newcommand{\FloatTok}[1]{\textcolor[rgb]{0.00,0.00,0.81}{#1}}
\newcommand{\FunctionTok}[1]{\textcolor[rgb]{0.00,0.00,0.00}{#1}}
\newcommand{\ImportTok}[1]{#1}
\newcommand{\InformationTok}[1]{\textcolor[rgb]{0.56,0.35,0.01}{\textbf{\textit{#1}}}}
\newcommand{\KeywordTok}[1]{\textcolor[rgb]{0.13,0.29,0.53}{\textbf{#1}}}
\newcommand{\NormalTok}[1]{#1}
\newcommand{\OperatorTok}[1]{\textcolor[rgb]{0.81,0.36,0.00}{\textbf{#1}}}
\newcommand{\OtherTok}[1]{\textcolor[rgb]{0.56,0.35,0.01}{#1}}
\newcommand{\PreprocessorTok}[1]{\textcolor[rgb]{0.56,0.35,0.01}{\textit{#1}}}
\newcommand{\RegionMarkerTok}[1]{#1}
\newcommand{\SpecialCharTok}[1]{\textcolor[rgb]{0.00,0.00,0.00}{#1}}
\newcommand{\SpecialStringTok}[1]{\textcolor[rgb]{0.31,0.60,0.02}{#1}}
\newcommand{\StringTok}[1]{\textcolor[rgb]{0.31,0.60,0.02}{#1}}
\newcommand{\VariableTok}[1]{\textcolor[rgb]{0.00,0.00,0.00}{#1}}
\newcommand{\VerbatimStringTok}[1]{\textcolor[rgb]{0.31,0.60,0.02}{#1}}
\newcommand{\WarningTok}[1]{\textcolor[rgb]{0.56,0.35,0.01}{\textbf{\textit{#1}}}}
\usepackage{longtable,booktabs,array}
\usepackage{calc} % for calculating minipage widths
% Correct order of tables after \paragraph or \subparagraph
\usepackage{etoolbox}
\makeatletter
\patchcmd\longtable{\par}{\if@noskipsec\mbox{}\fi\par}{}{}
\makeatother
% Allow footnotes in longtable head/foot
\IfFileExists{footnotehyper.sty}{\usepackage{footnotehyper}}{\usepackage{footnote}}
\makesavenoteenv{longtable}
\usepackage{graphicx}
\makeatletter
\def\maxwidth{\ifdim\Gin@nat@width>\linewidth\linewidth\else\Gin@nat@width\fi}
\def\maxheight{\ifdim\Gin@nat@height>\textheight\textheight\else\Gin@nat@height\fi}
\makeatother
% Scale images if necessary, so that they will not overflow the page
% margins by default, and it is still possible to overwrite the defaults
% using explicit options in \includegraphics[width, height, ...]{}
\setkeys{Gin}{width=\maxwidth,height=\maxheight,keepaspectratio}
% Set default figure placement to htbp
\makeatletter
\def\fps@figure{htbp}
\makeatother
\setlength{\emergencystretch}{3em} % prevent overfull lines
\providecommand{\tightlist}{%
  \setlength{\itemsep}{0pt}\setlength{\parskip}{0pt}}
\setcounter{secnumdepth}{-\maxdimen} % remove section numbering
\ifluatex
  \usepackage{selnolig}  % disable illegal ligatures
\fi

\title{Midterm Project}
\author{Yifeng He}
\date{December 4, 2021}

\begin{document}
\maketitle

\hypertarget{abstract}{%
\subsection{Abstract}\label{abstract}}

This report aims to examine the relationship between employee attrition
and five predictors using logit multilevel model. The model indicates
that age, monthly income, and relationship satisfaction have negative
effects on probability of attrition, while distance from home and years
since last promotion have positive effects on attrition. Reasons for the
effects are mixed, and more considerations need to be included when
applied in real world situations.

\hypertarget{introduction}{%
\subsection{Introduction}\label{introduction}}

Employee attrition means the reduction of workforce because of
retirement, death, sickness, and relocation, etc. Attrition is a natural
process in companies that decreases the work force without much
management efforts. Sometimes, however, unpredicted attrition may result
in extra cost of continuing the project and training for new workers. To
minimize the cost and the possible lost, companies are constantly trying
to reduce the attrition. Some companies invest the process of hiring to
find the right people, some provide comfortable working environments to
employees, and some simply increase employees' salaries to increase
their willingness to stay in the company. All those methods will improve
attractiveness for employees, but attrition still happens.

This report will use multilevel model to investigate what factors affect
attrition and how they influence employees' decision. Based on the
analysis, this report will also propose some ways to address the
attrition problem.

\hypertarget{method}{%
\subsection{Method}\label{method}}

\hypertarget{data-processing}{%
\subsubsection{Data Processing}\label{data-processing}}

The dataset used in this report is from Kaggle: IBM HR Analytics
Employee Attrition \& Performance. The dataset includes responses of
attrition from 1,470 subjects and their work-related conditions like
total working years, salaries, and satisfaction, etc. After preliminary
analysis of the data, I found that the dataset is already well written
and cleaned, that it contains no useless information. Then I pick five
variables that I am most interested in and listed below:

\begin{longtable}[]{@{}
  >{\raggedright\arraybackslash}p{(\columnwidth - 2\tabcolsep) * \real{0.35}}
  >{\raggedright\arraybackslash}p{(\columnwidth - 2\tabcolsep) * \real{0.65}}@{}}
\toprule
Variables & Explanation \\
\midrule
\endhead
Age & Age of the employee \\
Distance From Home & Distance from company to home \\
Environment Satisfaction & Satisfaction of working environment with 4
levels \\
Monthly Income & Monthly income of employee \\
Total Working Years & Working years since the first career \\
Years Since Last Promotion & Working years since last promotion \\
\bottomrule
\end{longtable}

Next, I transferred the dataset into the long format that is suitable
for exploratory data analysis (EDA).

\hypertarget{exploratory-data-analysis}{%
\subsubsection{Exploratory Data
Analysis}\label{exploratory-data-analysis}}

The radar plot below, figure 1, shows the mean value of the employees
who want to leave (green) and to stay (pink) in five selected factors.
The outer circle is the largest value of each factor from the dataset.
From the plot, employees who choose not to leave have higher average
monthly income than those who decide to leave. The same pattern shows in
age, as older people have lower average attrition. Mean values for other
factors have much less difference.

Figure 1, average values for five factors

To further see the difference between attrition and not attrition group,
a comparative boxplot shows more information will help. Figure 2 shows
the range, first and third quartiles, and median values for all five
factors. Log(value) is used to improve the scale for the values. From
figure 2, it is clear that Monthly Income, Distance From Home, and Age
have the most distinguishable difference; Relationship Satisfaction and
Year Since Last Promotion seems to be the same.

\begin{Shaded}
\begin{Highlighting}[]
\CommentTok{\#group box plots}
\NormalTok{attri\_try1 }\OtherTok{\textless{}{-}} \FunctionTok{pivot\_longer}\NormalTok{(attr\_try, }
                           \AttributeTok{cols =} \FunctionTok{c}\NormalTok{(}\DecValTok{1}\NormalTok{, }\DecValTok{3}\NormalTok{, }\DecValTok{4}\NormalTok{, }\DecValTok{5}\NormalTok{, }\DecValTok{6}\NormalTok{),}
                           \AttributeTok{names\_to =} \StringTok{"Factors"}\NormalTok{,}
                           \AttributeTok{values\_to =} \StringTok{"value"}
\NormalTok{                           )}
\FunctionTok{ggplot}\NormalTok{(attri\_try1, }\FunctionTok{aes}\NormalTok{(}\AttributeTok{x=}\NormalTok{Factors, }\AttributeTok{y=}\FunctionTok{log}\NormalTok{(value), }\AttributeTok{fill=}\FunctionTok{factor}\NormalTok{(Attrition))) }\SpecialCharTok{+} 
    \FunctionTok{geom\_boxplot}\NormalTok{()}\SpecialCharTok{+}
  \FunctionTok{theme}\NormalTok{(}\AttributeTok{axis.text=}\FunctionTok{element\_text}\NormalTok{(}\AttributeTok{size=}\FloatTok{5.5}\NormalTok{, }\AttributeTok{face =} \StringTok{"bold"}\NormalTok{))}
\end{Highlighting}
\end{Shaded}

\begin{verbatim}
## Warning: Removed 581 rows containing non-finite values (stat_boxplot).
\end{verbatim}

\begin{center}\includegraphics[width=0.9\linewidth]{MA678-Midterm-Project_files/figure-latex/unnamed-chunk-3-1} \end{center}

Figure 2, boxplot

Figure 3 shows the relationship between attrition or not and monthly
income for all the job roles. It is noteworthy that different jobs react
nearly opposite to increased amount of monthly income. Sales executive,
healthcare representative, and research director show increasing trend
of attrition when monthly income increases, though the slope is small.
Other jobs show clear decrease of willingness for attrition when monthly
income increases.

Figure 3, monthly income vs.~attrition

Figure 4a shows the relationship between attrition and the distance from
work to home. Sales representative shows decreasing trend of attrition
when distance from home increases; other jobs show increasing attrition
trend when distance from home increases. Figure 4b reveals the
connection between the time since last promotion and employee attrition.
Sales representative, human resources, and lab technician have reduction
in attrition when last promotion time increase; other jobs show the
opposite effect that the longer the time from last promotion, the higher
the attrition, but the overall slopes are low.

Figure 4, distance from home and years from promotion vs.~attrition

\hypertarget{model-fitting}{%
\subsubsection{Model fitting}\label{model-fitting}}

With different job roles as categories, the best model to fit the data
is multilevel model. I choose 5 predictor variables, they are all
continuous variables, and attrition, a binary variable, as the outcome.
Since the outcome variable is binary, I will use logit multilevel model.
Below is the function:

The fixed effects of the model are shown in the table below:

\hypertarget{result}{%
\subsection{Result}\label{result}}

Based on the logit multilevel model, the formula can be concluded as
below:

Logit(attrition) = -0.36 - 0.03\emph{Age + 0.03} Distance - \ldots{}

For the above model, -0.36 means the probability of attrition for an
employee with average age, average income, and all the rest factors as
the average value. And for every one unit increase in age, the log odds
of attrition with decrease 0.03 on average. To make it easier to
understand, I will take the exp of the coefficients and transfer the
model as below:

Attrition = exp(-0.36 - 0.03\emph{Age + 0.03} Distance - \ldots) =0.70 *
0.97\^{}Age * 1.03 \^{} Distance * \ldots{}

The transferred model indicates that, with average age, distance from
home, monthly income, etc., the odds of attrition is 0.7. For one unit
increase in age, the odds of attrition will experience a multiplicative
effect of 0.97, which means that the probability of attrition will
decrease 3.0\%, when other predictors take the average values. Increase
in distance from home will raise the probability of attrition by 2.7\%;
monthly income will reduce the probability 0.007\%; relationship
satisfaction will decrease attrition 10\%; and year since last promotion
will increase the probability of attrition 3.6\%; all the changes of
predictor is one unit with other factors as average values.

From the model, it is clear that age, monthly income, and relationship
satisfaction have negative effects for the probability of attrition,
while distance from home and years since last promotion have positive
effect on attrition.

The binned residual plot below, Figure 5, shows that 95\% of the points
are within the boundaries, so the model fits good. Since I use logit
multilevel model, no other residual plots are needed.

\begin{Shaded}
\begin{Highlighting}[]
\NormalTok{b1}\OtherTok{\textless{}{-}}\FunctionTok{binnedplot}\NormalTok{(}\FunctionTok{fitted}\NormalTok{(fit1),}\FunctionTok{resid}\NormalTok{(fit1, }\AttributeTok{type =} \StringTok{"response"}\NormalTok{))}
\end{Highlighting}
\end{Shaded}

\begin{center}\includegraphics[width=0.9\linewidth]{MA678-Midterm-Project_files/figure-latex/unnamed-chunk-5-1} \end{center}

Figure 5, binned residual plot

\hypertarget{discussion}{%
\subsection{Discussion}\label{discussion}}

The transformed formula of the logit multilevel model shows that
increase in age, monthly income, and relationship satisfaction will lead
to low probability of attrition; on the opposite, distance from home and
years since last promotion will increase the likelihood of attrition.

Though all the factors influence the probability of attrition, the
effects are very different. Take monthly income as an example, one unit
of income increase only reduce the probability of attrition by 0.007\%.
However, the increase of monthly income is usually by thousands, so when
the income level increases, the reduction in attrition should be
relatively large. As shown in figure 1, the difference between monthly
income for attrition and not attrition groups are the most
distinguishable factor. The large gap between the medians in figure 2
also indicates that monthly income plays a significant role in affecting
employment attrition.

Other results are also easy to understand. Longer commute distance means
the commute time is longer, the cost of going to work is higher, thus
reduce the likelihood to stay in the company; longer time from promotion
demonstrate that there is no room for improvement and employees may
leave for other opportunities. Older people tend to stay in the company
for they are less likely to move; employees with good intimate
relationships usually satisfy with their environments. However, there is
no casual relationships between these predictors and the attrition
results. Unforeseen factors like family emergency or accident may also
affect the attrition. Any single predictor also can not predict the
results of attrition cause there is always complicated interactive
factors that lead to the result.

Future studies can include more factors into the model, and also examine
the interactive effects between these variables. However, this
correlated model never implicate true causational relationship between
attrition and any of the factors. The real-world situation is much more
complex than the model prediction, and time and efforts must be spend to
reduce employee attrition.

\hypertarget{citation}{%
\subsection{Citation}\label{citation}}

The citation is as below:

\hypertarget{appendix}{%
\subsection{Appendix}\label{appendix}}

\begin{Shaded}
\begin{Highlighting}[]
\CommentTok{\#scatter plots EDA}
\NormalTok{plot1 }\OtherTok{\textless{}{-}}\NormalTok{ age\_point }\OtherTok{\textless{}{-}} \FunctionTok{ggplot}\NormalTok{(}\AttributeTok{data=}\NormalTok{attrition)}\SpecialCharTok{+}
  \FunctionTok{geom\_point}\NormalTok{(}\FunctionTok{aes}\NormalTok{(}\AttributeTok{x=}\NormalTok{Age,}\AttributeTok{y=}\NormalTok{Attrition,}\AttributeTok{color=}\NormalTok{JobRole), }\AttributeTok{alpha =} \FloatTok{0.3}\NormalTok{)}\SpecialCharTok{+}
  \FunctionTok{geom\_smooth}\NormalTok{(}\FunctionTok{aes}\NormalTok{(}\AttributeTok{x=}\NormalTok{Age,}\AttributeTok{y=}\NormalTok{Attrition,}\AttributeTok{color=}\NormalTok{JobRole), }\AttributeTok{method =} \StringTok{"lm"}\NormalTok{, }\AttributeTok{se =} \ConstantTok{FALSE}\NormalTok{)}\SpecialCharTok{+}
  \FunctionTok{theme\_ipsum}\NormalTok{()}\SpecialCharTok{+}
  \FunctionTok{theme}\NormalTok{(}\AttributeTok{legend.position =} \StringTok{"none"}\NormalTok{)}

\NormalTok{plot2 }\OtherTok{\textless{}{-}} \FunctionTok{ggplot}\NormalTok{(}\AttributeTok{data=}\NormalTok{attrition)}\SpecialCharTok{+}
  \FunctionTok{geom\_point}\NormalTok{(}\FunctionTok{aes}\NormalTok{(}\AttributeTok{x=}\NormalTok{DistanceFromHome,}\AttributeTok{y=}\NormalTok{Attrition,}\AttributeTok{color=}\NormalTok{JobRole), }\AttributeTok{alpha =} \FloatTok{0.3}\NormalTok{)}\SpecialCharTok{+}
  \FunctionTok{geom\_smooth}\NormalTok{(}\FunctionTok{aes}\NormalTok{(}\AttributeTok{x=}\NormalTok{DistanceFromHome,}\AttributeTok{y=}\NormalTok{Attrition,}\AttributeTok{color=}\NormalTok{JobRole), }\AttributeTok{method =} \StringTok{"lm"}\NormalTok{, }\AttributeTok{se =} \ConstantTok{FALSE}\NormalTok{)}\SpecialCharTok{+}
  \FunctionTok{theme\_ipsum}\NormalTok{()}\SpecialCharTok{+}
  \FunctionTok{theme}\NormalTok{(}\AttributeTok{legend.position =} \StringTok{"none"}\NormalTok{)}

\NormalTok{plot3 }\OtherTok{\textless{}{-}} \FunctionTok{ggplot}\NormalTok{(}\AttributeTok{data=}\NormalTok{attrition)}\SpecialCharTok{+}
  \FunctionTok{geom\_point}\NormalTok{(}\FunctionTok{aes}\NormalTok{(}\AttributeTok{x=}\NormalTok{MonthlyIncome,}\AttributeTok{y=}\NormalTok{Attrition,}\AttributeTok{color=}\NormalTok{JobRole), }\AttributeTok{alpha =} \FloatTok{0.3}\NormalTok{)}\SpecialCharTok{+}
  \FunctionTok{geom\_smooth}\NormalTok{(}\FunctionTok{aes}\NormalTok{(}\AttributeTok{x=}\NormalTok{MonthlyIncome,}\AttributeTok{y=}\NormalTok{Attrition,}\AttributeTok{color=}\NormalTok{JobRole), }\AttributeTok{method =} \StringTok{"lm"}\NormalTok{, }\AttributeTok{se =} \ConstantTok{FALSE}\NormalTok{)}\SpecialCharTok{+}
  \FunctionTok{theme\_ipsum}\NormalTok{()}

\FunctionTok{grid.arrange}\NormalTok{(plot1, plot2, plot3,}
             \AttributeTok{layout\_matrix =} \FunctionTok{rbind}\NormalTok{(}\FunctionTok{c}\NormalTok{(}\DecValTok{1}\NormalTok{,}\DecValTok{2}\NormalTok{),}
                                   \FunctionTok{c}\NormalTok{(}\DecValTok{1}\NormalTok{,}\DecValTok{2}\NormalTok{),}
                                   \FunctionTok{c}\NormalTok{(}\DecValTok{3}\NormalTok{,}\DecValTok{3}\NormalTok{))}
\NormalTok{             )}
\end{Highlighting}
\end{Shaded}

\begin{verbatim}
## `geom_smooth()` using formula 'y ~ x'
## `geom_smooth()` using formula 'y ~ x'
## `geom_smooth()` using formula 'y ~ x'
\end{verbatim}

\begin{center}\includegraphics[width=0.9\linewidth]{MA678-Midterm-Project_files/figure-latex/unnamed-chunk-6-1} \end{center}

\begin{Shaded}
\begin{Highlighting}[]
\NormalTok{plot4 }\OtherTok{\textless{}{-}} \FunctionTok{ggplot}\NormalTok{(}\AttributeTok{data=}\NormalTok{attrition)}\SpecialCharTok{+}
  \FunctionTok{geom\_point}\NormalTok{(}\FunctionTok{aes}\NormalTok{(}\AttributeTok{x=}\NormalTok{RelationshipSatisfaction,}\AttributeTok{y=}\NormalTok{Attrition,}\AttributeTok{color=}\NormalTok{JobRole), }\AttributeTok{alpha =} \FloatTok{0.3}\NormalTok{)}\SpecialCharTok{+}
  \FunctionTok{geom\_smooth}\NormalTok{(}\FunctionTok{aes}\NormalTok{(}\AttributeTok{x=}\NormalTok{TotalWorkingYears,}\AttributeTok{y=}\NormalTok{Attrition,}\AttributeTok{color=}\NormalTok{JobRole), }\AttributeTok{method =} \StringTok{"lm"}\NormalTok{, }\AttributeTok{se =} \ConstantTok{FALSE}\NormalTok{)}\SpecialCharTok{+}
  \FunctionTok{theme\_ipsum}\NormalTok{()}\SpecialCharTok{+}
  \FunctionTok{theme}\NormalTok{(}\AttributeTok{legend.position =} \StringTok{"none"}\NormalTok{)}

\NormalTok{plot5 }\OtherTok{\textless{}{-}} \FunctionTok{ggplot}\NormalTok{(}\AttributeTok{data=}\NormalTok{attrition)}\SpecialCharTok{+}
  \FunctionTok{geom\_point}\NormalTok{(}\FunctionTok{aes}\NormalTok{(}\AttributeTok{x=}\NormalTok{YearsSinceLastPromotion,}\AttributeTok{y=}\NormalTok{Attrition,}\AttributeTok{color=}\NormalTok{JobRole), }\AttributeTok{alpha =} \FloatTok{0.3}\NormalTok{)}\SpecialCharTok{+}
  \FunctionTok{geom\_smooth}\NormalTok{(}\FunctionTok{aes}\NormalTok{(}\AttributeTok{x=}\NormalTok{YearsSinceLastPromotion,}\AttributeTok{y=}\NormalTok{Attrition,}\AttributeTok{color=}\NormalTok{JobRole), }\AttributeTok{method =} \StringTok{"lm"}\NormalTok{, }\AttributeTok{se =} \ConstantTok{FALSE}\NormalTok{)}\SpecialCharTok{+}
  \FunctionTok{theme\_ipsum}\NormalTok{()}

\FunctionTok{grid.arrange}\NormalTok{(plot4, plot5,}
             \AttributeTok{layout\_matrix =} \FunctionTok{rbind}\NormalTok{(}\FunctionTok{c}\NormalTok{(}\DecValTok{1}\NormalTok{,}\DecValTok{1}\NormalTok{,}\DecValTok{2}\NormalTok{,}\DecValTok{2}\NormalTok{,}\DecValTok{2}\NormalTok{))}
\NormalTok{             )}
\end{Highlighting}
\end{Shaded}

\begin{verbatim}
## `geom_smooth()` using formula 'y ~ x'
## `geom_smooth()` using formula 'y ~ x'
\end{verbatim}

\begin{center}\includegraphics[width=0.9\linewidth]{MA678-Midterm-Project_files/figure-latex/unnamed-chunk-6-2} \end{center}

\begin{Shaded}
\begin{Highlighting}[]
\CommentTok{\# ggplot(attrition)+aes(x=Age,y=Attrition,color=JobRole)+geom\_jitter()+geom\_smooth(aes(x=Age,y=Attrition,color=JobRole), method = "lm", se = FALSE)}
\CommentTok{\# }
\CommentTok{\# ggplot(data=attrition)+}
\CommentTok{\#   geom\_point(aes(x=Age,y=Attrition,color=JobRole), alpha = 0.3)+}
\CommentTok{\#   theme\_ipsum()+}
\CommentTok{\#   geom\_jitter()+}
\CommentTok{\#   geom\_smooth(aes(x=Age,y=Attrition,color=JobRole),color="blue")}
\end{Highlighting}
\end{Shaded}

\begin{Shaded}
\begin{Highlighting}[]
\CommentTok{\#marginal distribution EDA}
\CommentTok{\# p \textless{}{-} ggplot(attrition, aes(x=MonthlyIncome,y=Attrition,color=JobRole)) +}
\CommentTok{\#         geom\_point()}
\CommentTok{\# }
\CommentTok{\# ggMarginal(p, type="density")}
\CommentTok{\# ggMarginal(p, type="histogram", fill = "slateblue", xparams = list(bins=10), margins = \textquotesingle{}x\textquotesingle{})}
\CommentTok{\# ggMarginal(p, margins = \textquotesingle{}x\textquotesingle{}, color="purple", size=4)}

\CommentTok{\#conditional density plot}
\CommentTok{\# cdplot(factor(Attrition) \textasciitilde{} Age, data=attrition)}

\CommentTok{\#violin plots}
\CommentTok{\# ggplot(attrition, aes(x=JobRole, y=Age, fill=factor(Attrition))) +}
\CommentTok{\#     geom\_violin() +}
\CommentTok{\#     scale\_fill\_viridis(discrete = TRUE, alpha=0.6, option="A") +}
\CommentTok{\#     theme\_ipsum() +}
\CommentTok{\#     theme(}
\CommentTok{\#       legend.position="none",}
\CommentTok{\#       plot.title = element\_text(size=11)}
\CommentTok{\#     ) +}
\CommentTok{\#     ggtitle("Violin chart") +}
\CommentTok{\#     xlab("")}
\CommentTok{\# }
\CommentTok{\# \#boxplot}
\CommentTok{\# ggplot(attrition, aes(x=JobRole, y=Age, fill=factor(Attrition))) +}
\CommentTok{\#     geom\_boxplot() +}
\CommentTok{\#     scale\_fill\_viridis(discrete = TRUE, alpha=0.6) +}
\CommentTok{\#     geom\_jitter(color="black", size=0.4, alpha=0.2) +}
\CommentTok{\#     theme\_ipsum() +}
\CommentTok{\#     theme(}
\CommentTok{\#       legend.position="none",}
\CommentTok{\#       plot.title = element\_text(size=11)}
\CommentTok{\#     ) +}
\CommentTok{\#     ggtitle("A boxplot with jitter") +}
\CommentTok{\#     xlab("")}
\end{Highlighting}
\end{Shaded}

\begin{Shaded}
\begin{Highlighting}[]
\CommentTok{\#other EDAs}
\CommentTok{\# ggplot(attrition, aes(x = \textasciigrave{}MonthlyIncome\textasciigrave{}, y = \textasciigrave{}JobRole\textasciigrave{}, fill = ..x..)) +}
\CommentTok{\#   geom\_density\_ridges\_gradient(scale = 3, rel\_min\_height = 0.01) +}
\CommentTok{\#   scale\_fill\_viridis(name = "Temp. [F]", option = "C") +}
\CommentTok{\#   labs(title = \textquotesingle{}Temperatures in Lincoln NE in 2016\textquotesingle{}) +}
\CommentTok{\#   theme\_ipsum() +}
\CommentTok{\#     theme(}
\CommentTok{\#       legend.position="none",}
\CommentTok{\#       panel.spacing = unit(0.1, "lines"),}
\CommentTok{\#       strip.text.x = element\_text(size = 8)}
\CommentTok{\#     )}
\CommentTok{\# }
\CommentTok{\# ggplot(attrition, aes(x=Age, y=Attrition, size = MonthlyIncome, color = JobRole)) +}
\CommentTok{\#     geom\_point(alpha=0.7)}
\end{Highlighting}
\end{Shaded}

\begin{quote}
\begin{quote}
\begin{quote}
\begin{quote}
\begin{quote}
\begin{quote}
\begin{quote}
9ae135952a0300875076d023246d7beba9bd26ab
\end{quote}
\end{quote}
\end{quote}
\end{quote}
\end{quote}
\end{quote}
\end{quote}

\end{document}
